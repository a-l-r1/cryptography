\documentclass[12pt,a4paper]{article}

\usepackage{amsmath}
\usepackage{amsfonts}
\usepackage{amssymb}
\usepackage{listings}

\XeTeXlinebreaklocale "zh"
\XeTeXlinebreakskip = 0pt plus 1pt

\usepackage{fontspec}
\setmainfont{Noto Sans CJK SC Regular}
\setmonofont{Noto Sans Mono CJK SC Regular}
\newfontfamily\titlefont[SizeFeatures={Size=20}]{Noto Sans CJK SC Regular}

\lstset{basicstyle=\ttfamily, language=Python}

\begin{document}

{
\titlefont
\begin{center}
密码学第九次实验报告
\end{center}
}

\section*{SHA-1 哈希算法}

\subsection*{原理}

  SHA-1 哈希算法是比较经典的一种哈希算法, 适用于数据的验证等需求. 哈希算法是单向算法, 给定原来的数据, 能够产生长度固定的哈希值. 
\newline
  SHA-1 哈希算法采用了 Merkle-Damgård 结构, 这种结构比较简单, 在理论上也能证明: 只要压缩函数 $ f $ 满足无碰撞性, 整个哈希函数也满足无碰撞性. 首先把消息按一定的规则进行填充, 然后把填充后的分组按顺序放入 $ f $ 函数, 输出作为下一轮执行 $ f $ 函数的参数之一. 第一轮没有上一轮执行 $ f $ 函数的结果, 就用初始向量 $ IV $ 代替. 其实, SHA-1 和 MD5 也比较相似. 
\newline
  SHA-1 哈希算法的 $ f $ 函数是通过不同的混淆和置换方式, 来达到相对安全的消息杂凑的. 其中有模 $ \mathrm{2^{32}} $ 加法、按位运算等方式. 
\newline
  实际上, SHA-1 已经能够相对容易地产生碰撞了, 并且由于硬件计算能力的日益强大和算法的进步, 攻击正在变得越来越现实. 因此, SHA-1 已经可以认为被废弃了. 

\subsection*{伪代码}

\subsubsection*{$ f $ 函数}

这里 $ f $ 函数中使用的加法, 结果都会取最后 32 位, $ <<< $ 代表对 32 位整数进行的循环移位运算. 
\newline
\lstinline{def _f(block: bytes, last_result: bytes) -> bytes:}
\newline
\lstinline{    for i in range(16):}
\newline
\lstinline{        w[i] = block[4 * i: 4 * i + 4]} 按大端序转换成的整数
\newline
\lstinline{    for i in range(16, 80):}
\newline
\lstinline{        w[i] = }$(w_{i - 3} \oplus w_{i - 8} \oplus w_{i - 14} \oplus w_{i - 16}) <<< 1 $
\newline
\lstinline{    for i in range(5):}
\newline
\lstinline{        h[i] = last_result[4 * i: 4 * i + 4]} 按大端序转换成的整数
\newline
\lstinline{    a, b, c, d, e = h[0], h[1], h[2], h[3], h[4]}
\newline
\lstinline{    for i in range(80):}
\newline
\lstinline{        if }$ 0 \le i \le 19 $\lstinline{:}
\newline
\lstinline{            f = }$ (b \wedge c) \vee ((\neg b) \wedge d) $
\newline
\lstinline{            k = 0x5a827999}
\newline
\lstinline{        elif }$ 20 \le i \le 39 $\lstinline{:}
\newline
\lstinline{            f = }$ b \oplus c \oplus d $\lstinline{:}
\newline
\lstinline{            k = 0x6ed9eba1}
\newline
\lstinline{        elif }$ 40 \le i \le 59 $\lstinline{:}
\newline
\lstinline{            f = }$ (b \wedge c) \vee (b \wedge d) \vee (c \wedge d) $
\newline
\lstinline{            k = 0x8f1bbcdc}
\newline
\lstinline{        else:}
\newline
\lstinline{            f = }$ b \oplus c \oplus d $
\newline
\lstinline{            k = 0xca62c1d6}
\newline
\lstinline{        tmp = }$ (a <<< 5) + f + e + k + w_i $
\newline
\lstinline{        e = d}
\newline
\lstinline{        d = c}
\newline
\lstinline{        c = }$ b <<< 30 $
\newline
\lstinline{        b = a}
\newline
\lstinline{        a = tmp}
\newline
\lstinline{    h[0] = h[0] + a}
\newline
\lstinline{    h[1] = h[1] + b}
\newline
\lstinline{    h[2] = h[2] + c}
\newline
\lstinline{    h[3] = h[3] + d}
\newline
\lstinline{    h[4] = h[4] + e}
\newline
\lstinline{    return }各 \lstinline{h[i]} 按大端序转换成 4 个字节再按下标顺序拼接成的字节串

\subsubsection*{消息分组函数}

\lstinline{def _pad_and_iter_chunks(m: bytes) -> Iterable[bytes]:}
\newline
\lstinline{    result = m}
\newline
\lstinline{    result} 后面填充一个比特位 1
\newline
\lstinline{    result} 后面不断填充比特位 0, 直到 \lstinline{result} 的长度 $ l $ 满足 $ l \equiv 448 (\mod 512) $
\newline
\lstinline{    result} 后面增加以 64 位大端序整数表示的 $ l $
\newline
\lstinline{    return result} 按 512 字节分组后形成的若干组

\subsubsection*{Merkle–Damgård 结构哈希算法}

\lstinline{def merkle_damgard_wide_pipe(iv: bytes, blocks: Iterable[bytes], }
\newline
\lstinline{                                f: Callable[[bytes, bytes], bytes], }
\newline
\lstinline{                                block_length: int, }
\newline
\lstinline{                                output_length: int) -> bytes:}
\newline
\lstinline{    if }$ iv $ 的长度与 \lstinline{output_length} 不同:
\newline
\lstinline{        raise }长度错误
\newline
\lstinline{    result = iv}
\newline
\lstinline{    for block in blocks:}
\newline
\lstinline{        if block} 的长度与 \lstinline{block_length} 不同\lstinline{:}
\newline
\lstinline{            raise }长度错误
\newline
\lstinline{        result = f(block, result)}
\newline
\lstinline{        if result} 的长度与 \lstinline{block_length} 不同\lstinline{:}
\newline
\lstinline{            raise }长度错误
\newline
\lstinline{    return result}

\subsubsection*{SHA-1 算法}

\lstinline{def digest(m: bytes) -> bytes:}
\newline
\lstinline{    return hash_common.merkle_damgard_wide_pipe(}$ iv $\lstinline{, }
\newline
\lstinline{                                                     m} 生成的各个分组\lstinline{, }$ f $\lstinline{, }
\newline
\lstinline{                                                     block_length, }
\newline
\lstinline{                                                     output_length)}

\subsubsection*{HMAC 算法}

\lstinline{def hmac(m: bytes, k: bytes, f_digest: Callable[[bytes], bytes], }
\newline
\lstinline{          block_length: int) -> bytes:}
\newline
\lstinline{    if len(k) > block_length:}
\newline
\lstinline{        k = }$ f(k) $
\newline
\lstinline{    k = }$ k \parallel b - k $ 的长度那么多个 $ 0 $ 字节
\newline
\lstinline{    k_s1 = }$ k \oplus ipad $
\newline
\lstinline{    hash_s1 = f_digest(}$ k_s1 \parallel m $\lstinline{)}
\newline
\lstinline{    k_s2 = }$ k \oplus opad $
\newline
\lstinline{    hash_s2 = f_digest(}$ k_s2 \parallel hash\_s1 $\lstinline{)}
\newline
\lstinline{    return hash_s2}

\subsubsection*{SHA-1 HMAC 算法}

\lstinline{def hmac(m: bytes, k: bytes) -> bytes:}
\newline
\lstinline{    return hash_common.hmac(m, k, digest, block_length)}

\subsection*{分析}

  以下设消息长度为 $ l $. 

\subsubsection*{$ f $ 函数}

  由于 $ f $ 函数的输入规模和计算语句的执行次数恒定, 所以它的时空复杂度都是 $ \mathrm{O}(1) $.

\subsubsection*{消息分组函数}

  消息分组函数附加的数据长度是有上限的, 但是产生的分组与长度大致成正比关系. 因此易知它的时间复杂度为 $ \mathrm{O}(l) $. 由于需要的空间上限是固定的(最多 512 位), 空间复杂度为 $ \mathrm{O}(1) $. 

\subsubsection*{Merkle–Damgård 结构哈希算法}

  该结构的哈希算法的时间复杂度由 $ f $ 函数决定. 这里 $ f $ 函数时空复杂度都是 $ \mathrm{O}(1) $. 但是分组有 $ \mathrm{O}(l) $ 个, 所以时间复杂度是 $ \mathrm{O}(l) $. 由于数据之间没有依赖, 而且 $ f $ 函数的空间复杂度是 $ \mathrm{O}(1) $, 所以空间复杂度是 $ \mathrm{O}(1) $. 

\subsubsection*{SHA-1 算法}

  该算法就是 Merkle–Damgård 结构哈希算法的一个封装, 所以时间复杂度和空间复杂度也分别是 $ \mathrm{O}(l) $ 和 $ \mathrm{O}(1) $. 

\subsubsection*{HMAC 算法}

  设密钥的长度为 $ l_K $. 
\newline
  首先若密钥长度超过相应哈希算法的分组长度, 则密钥的哈希值就要被计算出来. 所以这里的时间复杂度是 $ \mathrm{O}(l_K) $, 空间复杂度是 $ \mathrm{O}(1) $. 之后, 密钥的长度就固定了. 对密钥的填充和与 $ ipad $ 或 $ opad $ 的异或时空复杂度都是 $ \mathrm{O}(1) $. 
\newline
  然后对消息和密钥拼接, 再求哈希值. 这里密钥经过处理后, 长度恒定, 所以这步的时间复杂度是 $ \mathrm{O}(l) $, 空间复杂度是 $ O(1) $. 
\newline
  之后的步骤数据规模恒定, 需要的时间和空间也恒定, 所以时空复杂度都是 $ \mathrm{O}(1) $. 
\newline
  所以算法总的时空复杂度分别是 $ \mathrm{O}(l_K + l) $ 和 $ \mathrm{O}(1) $. 

\subsubsection*{SHA-1 HMAC 算法}
  该算法是 HMAC 算法的一个封装, 所以时空复杂度和它相同, 也都分别是 $ \mathrm{O}(l_K + l) $ 和 $ \mathrm{O}(1) $. 

\subsection*{优化}

\subsubsection*{$ f $ 函数}

  $ i \in [0, 19] $ 时的逻辑函数是按位选择函数, $ i \in [40, 59] $ 时的逻辑函数是按位取多数函数. 这些函数都有多种变形, 有的利于通用处理器实现, 有的利于专用电路实现. 
\newline
  $ w_i $ 的计算在第 32 - 79 轮时可以优化成 $ w_i = (w_{i - 6} \oplus w_{i - 16} \oplus w_{i - 28} \oplus w_{i - 32}) <<< 2 $. 这种变换可以使得各操作保持 64 字节对齐, 并且把 $ w_i $ 向 $ w_{i - 3} $ 的依赖去掉了, 更有利于 SIMD 等向量指令集实现 SHA-1. 

\subsubsection*{消息分组函数}

  消息分组函数可以计算出要填充的消息末尾后, 用 Python 的迭代器实现, 只要迭代到消息末尾后, 再追加即可. 这样可以减少中间结果的内存占用, 也可以在 HMAC 算法中复用. 

\subsubsection*{Merkle-Damgård 结构哈希算法和 SHA-1 算法}

  这种算法虽然安全性较高, 但是好像数据依赖性也较高, 相对难以在并行性上优化. 但是可以通过优化 $ f $ 函数, 间接优化哈希算法. 

\subsubsection*{HMAC 算法和 SHA-1 HMAC 算法}

  由于 $ K^+ \oplus ipad $ 和 $ K^+ \oplus opad $ 是可以提前计算出来的, 而且如果下层的哈希算法是 Merkle-Damgård 结构的话, 可以提前把这块放入 $ f $ 函数计算结果, 并且把结果当作相应的新初始向量 $ iv' $, 所以 SHA-1 HMAC 算法可以这样优化. 同样地, 并行性也是一个问题, 但是提前计算出新的 $ iv' $ 能够稍微提高并行性, 而且能够提高计算短消息的 HMAC 的速度. 

\subsection*{测试}

  采用 \verb|sha1_test.py| 进行自动测试. 该文件大致上是选择 10000 个随机字节串作为消息, 10000 个随机字节串作为密钥, 把求出的结果与标准库中的 SHA-1 和 HMAC 算法进行比较. 如果有不同, 就打印出产生错误的消息和密钥, 否则打印出“\verb|test passed|”并退出. 
\newline
  运行了该文件多次, 都能通过测试. 所以可以认为算法没有问题. 

\section*{Hash 函数生日攻击}

\subsection*{原理}

  对 Hash 函数的生日攻击是把 Hash 函数看成输入随机输出也随机但对某个确定输入输出确定的函数. 这种假设也符合理想 Hash 函数的性质. 这里的生日攻击是找出一对消息 $ x $ 和 $ y $, $ s.t. H(x) = H(y) $. 其中 $ H(m) $ 为哈希函数. 
\newline
  可以通过概率论的知识得到, 若 Hash 函数是理想 Hash 函数, 攻击的代价是大约 $ 2^{\frac{n}{2}} $ 次 Hash 运算, 其中 $ n $ 是 Hash 函数结果的二进制位数. 

\subsection*{伪代码}

\lstinline{def find_collision_pair(}
\newline
\lstinline{    rand_bytes_generator_func: Callable[[bytes], bytes], }
\newline
\lstinline{    hash_func: Callable[[bytes], bytes]}
\newline
\lstinline{    ) -> Tuple[bytes, bytes]:}
\newline
\lstinline{    b}$_1$\lstinline{, b}$_2$ = 两对随机消息
\newline
\lstinline{    tries = 1}
\newline
\lstinline{    while }$ H(b_1) $\lstinline{ != }$ H(b_2) $\lstinline{ or }$ b_1 $\lstinline{ == }$ b_2 $\lstinline{:}
\newline
\lstinline{        b}$_1$\lstinline{, b}$_2$\lstinline{ = }两对随机消息
\newline
\lstinline{        tries += 1}
\newline
\lstinline{        if tries % 1024 == 0:}
\newline
\lstinline{            print(tries)}
\newline
\lstinline{    return b}$_1$\lstinline{, b}$_2$

\subsection*{分析}

  由于攻击的代价是大约 $ 2^{\frac{n}{2}} $ 次 Hash 运算, 而且每次 Hash 运算的时空复杂度都是 $ \mathrm{O}(1) $, 再加上每次 Hash 计算时没有数据依赖, 所以整个算法的时间复杂度为 $ \mathrm{O}(2^\frac{n}{2}) $, 空间复杂度为 $ \mathrm{O}(1) $. 

\subsection*{优化}

  其实该算法相当好并行, 所以可以并行计算, 这样可以线性地提高效率. 其实也可以先计算短消息, 再不断延长, 这样可以直接把已经算好的哈希作为新的初始向量, 提高计算效率. 
\newline
  实际上, 对已经有的哈希算法采用这种攻击不现实, 因为$ n $ 太大了. 在真正测试的时候, 采用了截断的哈希函数, 也就是把消息经过 SHA-1 哈希算法的结果取前两个字节. 这样 $ n = 16 $, 能够保证在可行的时间内找到一对哈希值相同的消息. 

\subsection*{测试}

  测试采用自动化测试, 文件为 \verb|hash_collision_test.py|. 文件会自动找出一对有冲突的消息, 并输出它们的内容和冲突的哈希值. 

\section*{总结}

\subsection*{SHA-1 哈希算法}

  这个算法是我第一次尝试写哈希算法. 这次写算法让我巩固了 Merkle-Damgård 结构的相关知识, 感觉更加明白了哈希的原理. 
\newline
  HMAC 算法让我更加明白了安全协议和在原语上构建协议的重要性, 它用简单的哈希算法构造出了相对复杂的原语, 但是安全性同样相当高. 

\subsection*{哈希函数生日攻击}

  哈希函数的生日攻击是对密码学函数的又一次攻击. 这次攻击我感觉很有意思, 能够用数学原理把攻击变成实用. 

\subsection*{算法分析与优化}

  这次对算法的优化也是有相当重要的地位的. 比如对 SHA-1 函数的优化, 可以让它更适合在各种场景应用. 对 HMAC 算法的优化, 是基于对算法的深刻理解. 

\subsection*{系统设计与维护}

  这次实验也是需要一定的系统设计的. 在对哈希函数的设计中, 可以把算法的结构和具体的 $ f $ 函数分开. 在对 HMAC 算法的设计中, 可以把哈希函数的 $ f $ 函数设计成可以改变初始向量 $ iv $, 这样就可以方便优化. 

\subsection*{对课程的建议}

  感觉以后可以把任务再减少一点, 因为我实在写不完了……写基本要求都要花很大劲去写. 不过现在的要求更加贴近原理, 感觉这样更好一些. 

\subsection*{总结}

  这次哈希算法实验感觉不错, 让我更加明白了哈希算法的原理. 我会努力面对以后的挑战. 


\end{document}
