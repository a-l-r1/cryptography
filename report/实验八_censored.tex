\documentclass[12pt,a4paper]{article}

\usepackage{fontspec}

\usepackage{ctex}

\usepackage{amsmath}
\usepackage{amsfonts}
\usepackage{amssymb}

\usepackage{indentfirst}
\usepackage[ruled]{algorithm2e}
\usepackage{graphviz}

\setmainfont{Roboto}
\setmonofont{Noto Sans Mono CJK SC Regular}
\setCJKmainfont{Noto Sans CJK SC Regular}
\setmathrm{Latin Modern Math}
\setmathtt{Latin Modern Mono}

\renewcommand{\algorithmcfname}{算法}

\SetKw{Continue}{continue}
\SetKw{Break}{break}

\SetKwInOut{KwInput}{输入}
\SetKwInOut{KwOutput}{输出}
\SetKwInOut{KwData}{数据}
\SetKwData{KwResult}{结果}

\zihao{5}



\usepackage{listings}

\lstset{basicstyle=\ttfamily, language=Python}

\begin{document}

{
\zihao{-2}

\begin{center}
密码学第八次实验报告
\end{center}
}

\section*{椭圆曲线相关算法}

\subsection*{原理}

  椭圆曲线是定义在射影平面中的一种曲线. 在密码学中能够应用的椭圆曲线的方程形式为
\begin{align*}
y^2 = x^3 + ax + b \quad (4 a^3 + 27 b^2 \neq 0)
\end{align*}
\noindent
  这种曲线原来是定义在 $ \mathbb{R} $ 上的, 但是也可以扩充到有限域 $ F $ 上. 这样更适合密码学应用. 
\newline
  由于椭圆曲线定义在射影平面上, 在普通平面上表示会丢失射影平面上的无穷远点, 所以还要定义一个无穷远点 $ P_{\infty} $. 
\newline
  定义椭圆曲线在同一条直线上的三个点之和为 $ P_{\infty} $. 可以证明出椭圆曲线上所有点关于这种加法形成一个 Abel 群, 加法单位元为 $ P_{\infty} $. 同时, 可以定义 $ 2P $ 为过点 $ P $ 的切线与椭圆曲线的另一个交点. 若无交点, 定义 $ 2P $ 为 $ P_{\infty} $. 这样, 就可以定义椭圆曲线上点的数乘. 
\newline
  椭圆曲线上点的加法可以推导出公式. 同样的, 求某个点左乘 $ 2 $ 也可以推导出公式. 这样, 椭圆曲线上点的加法和数乘都可以推导出公式, 这就把几何操作转化成了代数运算. 
\newline
  椭圆曲线有以下困难问题: 已知 $ P $ 和 $ kP (k \in \mathbb{Z} ) $, 求 $ k $. 该问题被称作椭圆曲线上的离散对数问题, 可以用来构造椭圆曲线上的公钥密码体制.

\subsection*{伪代码}

\subsubsection*{点的加法}

\lstinline{def __add__(self, other):}
\newline
\lstinline{    if self == }$ P_{\infty} $\lstinline{ and other == }$ P_{\infty} $\lstinline{:}
\newline
\lstinline{        return }$ P_{\infty} $
\newline
\lstinline{    if self == }$ P_{\infty} $\lstinline{:}
\newline
\lstinline{        return other}
\newline
\lstinline{    if other == }$ P_{\infty} $\lstinline{:}
\newline
\lstinline{        return self}
\newline
\lstinline{    if }$ x_{self} $\lstinline{ != }$ x_{other} $\lstinline{:}
\newline
\lstinline{        delta = }$ \frac{y_{other} - y_{self}}{x_{other} - x_{self}} $
\newline
\lstinline{        result.x = }$ {delta}^2 - x_{self} - x_{other} $
\newline
\lstinline{        result.y = }$ - y_{self} + delta * (x_{self} - x_{result}) $
\newline
\lstinline{    else:}
\newline
\lstinline{        if }$ y_{self} $\lstinline{ == }$ y_{other} $\lstinline{:}
\newline
\lstinline{        intermediate = }$ \frac{3x_{self}^2 + a}{2y_{self}} $
\newline
\lstinline{        result.x = }$ {intermediate}^2 - 2x_{self} $
\newline
\lstinline{        result.y = }$ {intermediate}(x_{self} - x_{result}) - y_{self} $
\newline
\lstinline{    else:}
\newline
\lstinline{    return }$ P_{\infty} $

\subsubsection*{点的数乘运算}

\lstinline{def __rmul__(self, other):}
\newline
\lstinline{    if other < 0:}
\newline
\lstinline{        raise }参数错误
\newline
\lstinline{    if other == 0:}
\newline
\lstinline{        return }$ P_{\infty} $
\newline
\lstinline{    if other == 1:}
\newline
\lstinline{        return self}
\newline
\lstinline{    if other == 2:}
\newline
\lstinline{        return self + self}
\newline
\lstinline{    curr_item = None}
\newline
\lstinline{    result = }$ P_{\infty} $
\newline
\lstinline{    for other} 从低到高的每一位\lstinline{:}
\newline
\lstinline{        if curr_item is None:}
\newline
\lstinline{            curr_item = self}
\newline
\lstinline{        else:}
\newline
\lstinline{            curr_item = 2 * curr_item}
\newline
\lstinline{        if }该位为 $1$\lstinline{:}
\newline
\lstinline{            result += curr_item}
\newline
\lstinline{    return result}

\subsection*{分析}

\subsubsection*{点的加法}

  很显然, 对点的加法的各种情况, 时空复杂度为 $ \mathrm{O}(1) $.

\subsubsection*{点的数乘运算}
「」
  设要乘的数为 $ m $.
\newline
  由于要遍历 $ m $ 的每一位, 所以易知时间复杂度为 $ \mathrm{O}(\log m) $. 每步之间没有数据关联, 所以空间复杂度为 $ O(1) $.

\subsection*{优化}

\subsubsection*{点的加法}

  实际上, 可能可以通过对点的加法的代数性质, 对点的加法进行优化. 我在网络上也找到了 NIST 的一种优化方法. 

\subsubsection*{点的数乘运算}

  对点的数乘运算, 由于前后数据相关性较大, 没有很好的优化方法. 但是可以借用模平方乘算法的优化方式(如加法链)来优化. 

\section*{椭圆曲线上的 Diffie-Hellman 密钥交换协议}

\subsection*{原理}

  Diffie-Hellman 密钥交换协议是基于离散对数问题求解困难性的. 椭圆曲线上的 Diffie-Hellman 密钥交换协议是基于椭圆曲线上离散对数问题求解困难性的.
\newline
  密钥交换双方首先生成 $ X_A $, $ X_B $ 作为私钥, 然后生成公钥 $ Y_A = X_A P $, $ Y_B = X_B P $, 其中 $ P $ 为椭圆曲线上规定好的基点. 收到对方的公钥时, 能够得到最终的密钥 $ Y = X_B Y_A P = X_A Y_B P $, 完成密钥交换. 

\subsection*{伪代码}

\subsubsection*{密钥生成算法}

\lstinline{def gen_key() -> Tuple[fp, ECPoint]:}
\newline
\lstinline{    x = }$ (1, n) $ 上的随机整数
\newline
\lstinline{    return x, x}$ P $

\subsubsection*{密钥获取算法}

\lstinline{def retrieve_key(sk: fp, p_: ECPoint) -> ECPoint:}
\newline
\lstinline{    return }$ {sk} \cdot P $

\subsection*{分析}

  由于所有算法中数据的上界已经给定, 且各子算法的时空复杂度已知, 所以两个算法的时空复杂度都是 $ \mathrm{O}(1) $. 

\subsection*{测试}

  测试模块为 \verb|diffie_hellman.py|. 主要的测试功能是测试密钥生成和接收算法及其正确性, 能够测试通过. 

\subsubsection*{优化}

  实际上, 两种算法比较简单, 我看不出明显的优化空间. 但是, 可以通过对子算法的优化, 来间接地优化两种算法. 

\section*{椭圆曲线上的 ElGamal 公钥密码体制}

\subsection*{原理}

  椭圆曲线上的 ElGamal 公钥加密算法也是基于椭圆曲线上的离散对数问题, 并且与常规的 ElGamal 公钥密码体制类似. 
\newline
  首先约定一条 $ F_p $ 上的椭圆曲线 $ E_p(a, b) $, 它的一个生成元 $ G $, 以及一个不超过 $ p $ 的数 $ n $.  

\subsubsection*{密钥生成算法}

  Alice 首先选择 $ (1, p) $ 上的一个随机数 $ d $, 把 $ d $ 作为私钥, $ Q = d P $ 作为公钥. 

\subsubsection*{加密算法}

  Bob 把满足 $ 1 \le m \le n $ 的消息 $ m $ 表示成 $ F_p $ 上的元素, 使用的字母不变. 然后他在 $ [1, n - 1] $ 内选择一个随机数 $ k $, 计算 $ C_1 = k P $. 然后计算 $ (x_2, y_2) = k Q $, 若 $ x_2 = 0 $, 则重新生成 $ k $, 重新计算. 然后计算 $ C_2 = m x_2 $, 并传送密文 $ (C_1, C_2) $ 给 Alice.

\subsubsection*{解密算法}

  Alice 使用私钥 $ d $, 计算 $ D = d C_1 $, 再计算 $ F_p $ 中 $ D $ 的横坐标 $ x_2 $ 的逆元 $ x_2^{-1} $. 然后通过 $ m = C_2 x_2^{-1} $ 恢复出明文 $ m $. 

\subsection*{伪代码}

\subsubsection*{密钥生成算法}

\lstinline{def gen_key() -> Tuple[fp, ECPoint]:}
\newline
\lstinline{    x = }$ (1, n) $ 中的一个随机数
\newline
\lstinline{    return }$ (x $\lstinline{, }$ x G) $

\subsubsection*{加密算法}

\lstinline{def encrypt(p_: int, pk: ECPoint) -> Tuple[ECPoint, int]:}
\newline
\lstinline{    if p_ >= n or p < 0:}
\newline
\lstinline{        raise }参数错误
\newline
\lstinline{    while True:}
\newline
\lstinline{        k = }$ [1, n - 1] $ 中的一个随机数
\newline
\lstinline{        x}$_1$\lstinline{ = }$ k G $
\newline
\lstinline{        x}$_2$\lstinline{ = }$ k \cdot pk $
\newline
\lstinline{        if x}$_2$ 的横坐标\lstinline{ == 0:}
\newline
\lstinline{            continue}
\newline
\lstinline{        c = }$ p_ \cdot x_2 $ 的横坐标
\newline
\lstinline{      return }$ (x_1, c) $

\subsubsection*{解密算法}
\lstinline{def decrypt(c: Tuple[ECPoint, int], sk: fp) -> int:}
\newline
\lstinline{    x}$_1$\lstinline{ = c[0]}
\newline
\lstinline{    c_int = c[1]}
\newline
\lstinline{    if c_int >= n or c_int < 0:}
\newline
\lstinline{         raise }解密出错
\newline
\lstinline{    x}$_2$\lstinline{ = }$ sk \cdot x_1$
\newline
\lstinline{    m = }$c\mathit{\_}int \cdot x_2$ 的横坐标
\newline
\lstinline{    return }$ m $

\subsection*{分析}

  由于数据的上界已知, 各子算法的时空复杂度也已知, 所以时空复杂度为 $ \mathrm{O}(1) $. 

\subsection*{测试}

  由于测试要求使用文件, 所以采用了文件加解密方式进行测试. 经过测试, 恢复的明文文件和原明文文件相同. 测试模块为 \verb|elgamal_test.py|.

\subsection*{优化}

\subsubsection*{主要算法}

  其实对主要算法, 我也没想出很好的方式来优化. 但是实际上同样也可以通过优化子算法来优化算法的效率. 

\subsubsection*{文件加解密算法}

  对文件加解密算法, 可以利用处理器的并行性来优化, 也可以提前算出需要的中间值. 

\section*{总结}

\subsection*{椭圆曲线相关算法}

  这些算法主要是对书上的公式的理解, 以及区分好各种特殊情况, 尤其是关于 $ P_{\infty} $ 的情况. 还有一种特殊情况值得注意: 自己与自己相加(或者说乘以 $ 2 $). 

\subsection*{椭圆曲线上的 Diffie-Hellman 密钥交换协议}

  这一部分也主要是对书上公式的理解. 但是这一部分主要是一个协议原语, 所以能实现的主要是一方, 但测试时要测试双方, 这一点要注意. 

\subsection*{椭圆曲线上的 ElGamal 公钥密码体制}

  这部分其实是以 Diffie-Hallman 密钥交换协议作为基础的. 其实该密码体制最重要的是它的随机性和密文扩张, 所以在与文件配合时要注意, 保存的文件会更大, 而且具有随机性. 

\subsection*{算法评估与优化}

  对这次实验里的算法, 我其实没有想到太好的优化方法. 但是, 我感觉真正的优化应该去掉抽象, 把底层的椭圆曲线算法和顶层的算法结合起来, 虽然这只是一个直觉. 

\subsection*{系统设计与维护}

  这次实验的完成, 实际上多少考验着系统设计能力. 椭圆曲线上点的算法, 实际上可以从点坐标所在的数域中抽象出来, 是点坐标的四则运算. 这时, 就可以使用运算符重载的方法, 来把对点坐标的运算抽象地写出来. 
\newline
  同时, 对于下层的 $ F_p $ 中的元素, 肯定要抽象成类. 把 $ p $ 固定, 会使得该类并不能可移植, 因此, 一种解决办法是给一个函数, 输入 $ p $, 返回表示 $ F_p $ 中元素的类. 这样就类似于元类(实例是类的类), 不过这里是生成类的函数. 
\newline
  对于椭圆曲线的各种参数, 可以把参数都封装到一个单独的模块里. 这里当然也得在那个模块中声明椭圆曲线对应的 $ F_p $. 声明完以后, 关于椭圆曲线的密码学原语只要导入对应的模块, 就能使用相应的椭圆曲线了. 
\newline
  对文件的 ECB 模式加解密, 也可以考虑把负责 ECB 模式的代码抽象出来, 形成一个函数. 输入的时候把输入和输出文件名、分块加密或解密函数、加密或解密时相应分块的长度当做函数参数输入进去, 负责 ECB 模式的函数就可以执行对文件的加解密了. 这样能减小重复代码的量, 也有利于代码的可移植性, 更好地实现和测试其它加密算法. 
\newline
  最终, 负责加解密的上层代码基本上不超过 20 行, 而且跟课件上的说明差不多是一致的. 这样极大地方便了开发, 也明显降低了 bug 发生的概率. 感觉这个实验其实比较需要总体的系统设计, 因为我想了两个小时, 才把大致的系统构架图想明白, 之后才开始编程. 

\subsection*{对课程的建议}

  感觉这次密码学实验课程终于回到正轨了, 不再出难死人的大整数题了……这次我感觉思路比较正, 就是考对课上内容的理解和熟练应用, 以及系统的合理设计, 其实并没有什么太刁钻或太细节的地方. 感觉以后密码学实验其实应该把系统设计的良好程度放入总评, 这样也对得起我们想出良好的系统结构的努力. 
\newline
  但是这次还有一个明显的缺陷, 就是我不知道怎么实现高级要求.我不知道国家密码局批准了什么样的随机数生成器. 我觉得我无法写出合规的 SM2 算法, 所以我只好放弃高级要求了. 同样地, 关于 SM9 算法, 我确实在国家密码局网站找到了官方资料, 但是英文文献实在太繁, 再加上我估计自己看不懂且需要一定学术水平, 于是也放弃了. 因此我在找资料上白白浪费了两个多小时, 却没有结果, 我感觉很可惜. 所以我有一个很冒昧的想法: 希望在布置实验任务以前, 能先评估一下可行性. 

\subsection*{总结}

  这次实验感觉路子比较正, 也很能考验系统设计能力. 当然, 我也感觉很有意思. 但是, 我感觉非常可惜的一点, 是我白白花了两个小时却找不到合适的资料. 所以希望以后老师或者助教学长们能帮我们找一下, 哪怕找最必需的也好. 

\end{document}
