\documentclass[12pt,a4paper]{article}

\usepackage{fontspec}

\usepackage{ctex}

\usepackage{amsmath}
\usepackage{amsfonts}
\usepackage{amssymb}

\usepackage{indentfirst}
\usepackage[ruled]{algorithm2e}
\usepackage{graphviz}

\setmainfont{Roboto}
\setmonofont{Noto Sans Mono CJK SC Regular}
\setCJKmainfont{Noto Sans CJK SC Regular}
\setmathrm{Latin Modern Math}
\setmathtt{Latin Modern Mono}

\renewcommand{\algorithmcfname}{算法}

\SetKw{Continue}{continue}
\SetKw{Break}{break}

\SetKwInOut{KwInput}{输入}
\SetKwInOut{KwOutput}{输出}
\SetKwInOut{KwData}{数据}
\SetKwData{KwResult}{结果}

\zihao{5}



\usepackage{listings}

\lstset{basicstyle=\ttfamily, language=Python}

\begin{document}

{
\zihao{2}
\begin{center}
密码学第八次实验报告
\end{center}
}

\section*{椭圆曲线相关算法}

\subsection*{原理}

椭圆曲线是定义在射影平面中的一种曲线. 在密码学中能够应用的椭圆曲线的方程形式为

\begin{gather*}
y^2 = x^3 + ax + b \quad (4 a^3 + 27 b^2 \neq 0)
\end{gather*}

这种曲线原来是定义在 $ \mathbb{R} $ 上的, 但是也可以扩充到有限域 $ F $ 上. 这样更适合密码学应用. 

由于椭圆曲线定义在射影平面上, 在普通平面上表示会丢失射影平面上的无穷远点, 所以还要定义一个无穷远点 $ P_{\infty} $. 

定义椭圆曲线在同一条直线上的三个点之和为 $ P_{\infty} $. 可以证明出椭圆曲线上所有点关于这种加法形成一个 Abel 群, 加法单位元为 $ P_{\infty} $. 同时, 可以定义 $ 2P $ 为过点 $ P $ 的切线与椭圆曲线的另一个交点. 若无交点, 定义 $ 2P $ 为 $ P_{\infty} $. 这样, 就可以定义椭圆曲线上点的数乘. 

椭圆曲线上点的加法可以推导出公式. 同样的, 求某个点左乘 $ 2 $ 也可以推导出公式. 这样, 椭圆曲线上点的加法和数乘都可以推导出公式, 这就把几何操作转化成了代数运算. 

椭圆曲线有以下困难问题: 已知 $ P $ 和 $ kP (k \in \mathbb{Z} ) $, 求 $ k $. 该问题被称作椭圆曲线上的离散对数问题, 可以用来构造椭圆曲线上的公钥密码体制.

\subsection*{伪代码}

% NOTE: have to remove titles because floating algorithms going everywhere and I have no time to fix this

\begin{algorithm}[H]
\caption{椭圆曲线上点的加法算法}
\KwInput{椭圆曲线上的两个点 $ A $ 和 $ B $}
\KwOutput{椭圆曲线上的点 $ C $}
\KwData{椭圆曲线参数 $ a, b $}

\If{$ A = P_\infty $ and $ B = P\infty$}{
    \Return{$ P_\infty $}
}
\If{$ A = P_\infty $}{
    \Return{$ B $}
}
\If{$ B = P_\infty $}{
    \Return{$ A $}
}
\eIf{$ x_A \neq x_B $}{
    $ \delta = \frac{y_B - y_A}{x_B - x_A} $
    
    $ x_C = \delta^2 - x_A - x_B $
    
    $ y_C = -y_A + \delta (x_A - x_C) $
    
    \Return{$ C $}
}
{
    \eIf{$ y_A = y_B $}{
        $ \delta = \frac{3 x_A^2 + a}{2 y_A} $
        
        $ x_C = \delta^2 - 2 x_A $
        
        $ y_C = -y_A + \delta (x_A - x_C) $
        
        \Return{$ C $}
    }
    {
        \Return{$ P_\infty $}
    }
}

\end{algorithm}

\begin{algorithm}[H]
\caption{椭圆曲线上点的数乘运算}
\KwInput{整数 $ n $,椭圆曲线上的点 $ P $}
\KwOutput{结果 $ R $}
\KwData{椭圆曲线参数 $ a, b $}

\If{$ n < 0 $}{
    抛出参数错误
}
\If{$ n = 0 $}{
    \Return{$ P_\infty $}
}
\If{$ n = 1 $}{
    \Return{$ P $}
}
\If{$ n = 2 $}{
    \Return{$ P + P $}
}
curr\_factor $ \leftarrow $ null
\For{$ n $ 从低到高的每个二进制位}{
    \eIf{curr\_factor $ = $ null}{
        curr\_factor $ \leftarrow $ self
    }
    {
        curr\_factor $ \leftarrow 2 * $ curr\_factor
    }
    \If{该位为 $ 1 $}{
        result $ \leftarrow $ result $ + $ curr\_factor
    }
}
\end{algorithm}

\subsection*{分析}

\subsubsection*{点的加法}

很显然, 对点的加法的各种情况, 时空复杂度为 $ \mathrm{O}(1) $.

\subsubsection*{点的数乘运算}

设要乘的数为 $ n $.

由于要遍历 $ n $ 的每一位, 所以易知时间复杂度为 $ \mathrm{O}(\log n) $. 每步之间没有数据关联, 所以空间复杂度为 $ O(1) $.

\subsection*{优化}

\subsubsection*{点的加法}

实际上, 可能可以通过对点的加法的代数性质, 对点的加法进行优化. 我在网络上也找到了 NIST 的一种优化方法. 

\subsubsection*{点的数乘运算}

对点的数乘运算, 由于前后数据相关性较大, 没有很好的优化方法. 但是可以借用模平方乘算法的优化方式(如加法链)来优化. 

\section*{椭圆曲线上的 Diffie-Hellman 密钥交换协议}

\subsection*{原理}

Diffie-Hellman 密钥交换协议是基于离散对数问题求解困难性的. 椭圆曲线上的 Diffie-Hellman 密钥交换协议是基于椭圆曲线上离散对数问题求解困难性的.

密钥交换双方首先生成 $ X_A $, $ X_B $ 作为私钥, 然后生成公钥 $ Y_A = X_A P $, $ Y_B = X_B P $, 其中 $ P $ 为椭圆曲线上规定好的基点. 收到对方的公钥时, 能够得到最终的密钥 $ Y = X_B Y_A P = X_A Y_B P $, 完成密钥交换. 

\subsection*{伪代码}

\subsubsection*{密钥生成算法}

\begin{algorithm}[H]
\caption{Diffie-Hallman 密钥交换协议密钥生成算法}
\KwInput{无}
\KwOutput{有限域上的数 $ n $,椭圆曲线上的点 $ p $}
\KwData{椭圆曲线的参数 $ a, b $, 它的一个生成元(同时也是系统公开参数)$ G $}

$ n \leftarrow (1, n) $ 上的随机整数
$ p \leftarrow n * G $
\Return{$ n, p $}
\end{algorithm}

\subsubsection*{密钥获取算法}

\begin{algorithm}[H]
\caption{Diffie-Hallman 密钥交换协议密钥获取算法}
\KwInput{己方私钥 $ x $,获取到的数据 $ P $}
\KwOutput{椭圆曲线上的点 $ P' $}

$ P' \leftarrow x P $

\Return{$ P' $}
\end{algorithm}

\subsection*{分析}

由于所有算法中数据的上界已经给定, 且各子算法的时空复杂度已知, 所以两个算法的时空复杂度都是 $ \mathrm{O}(1) $. 

\subsection*{测试}

测试模块为 \verb|diffie_hellman.py|. 主要的测试功能是测试密钥生成和接收算法及其正确性, 能够测试通过. 

\subsubsection*{优化}

实际上, 两种算法比较简单, 我看不出明显的优化空间. 但是, 可以通过对子算法的优化, 来间接地优化两种算法. 

\section*{椭圆曲线上的 ElGamal 公钥密码体制}

\subsection*{原理}

椭圆曲线上的 ElGamal 公钥加密算法也是基于椭圆曲线上的离散对数问题, 并且与常规的 ElGamal 公钥密码体制类似. 

首先约定一条 $ F_p $ 上的椭圆曲线 $ E_p(a, b) $, 它的一个生成元 $ G $, 以及一个不超过 $ p $ 的数 $ n $.  

\subsubsection*{密钥生成算法}

Alice 首先选择 $ (1, p) $ 上的一个随机数 $ d $, 把 $ d $ 作为私钥, $ Q = d P $ 作为公钥. 

\subsubsection*{加密算法}

Bob 把满足 $ 1 \le m \le n $ 的消息 $ m $ 表示成 $ F_p $ 上的元素, 使用的字母不变. 然后他在 $ [1, n - 1] $ 内选择一个随机数 $ k $, 计算 $ C_1 = k P $. 然后计算 $ (x_2, y_2) = k Q $, 若 $ x_2 = 0 $, 则重新生成 $ k $, 重新计算. 然后计算 $ C_2 = m x_2 $, 并传送密文 $ (C_1, C_2) $ 给 Alice.

\subsubsection*{解密算法}

Alice 使用私钥 $ d $, 计算 $ D = d C_1 $, 再计算 $ F_p $ 中 $ D $ 的横坐标 $ x_2 $ 的逆元 $ x_2^{-1} $. 然后通过 $ m = C_2 x_2^{-1} $ 恢复出明文 $ m $. 

\subsection*{伪代码}

\begin{algorithm}[H]
\caption{ElGamal 公钥密码体制密钥生成算法}
\KwInput{无}
\KwOutput{私钥 $ n $,公钥 $ P $}
\KwData{椭圆曲线的参数 $ a, b $,它的一个生成元 $ G $}

$ n \leftarrow (1, n) $ 中的随机数

$ P \leftarrow n G $

\Return{$ n, P $}
\end{algorithm}

\begin{algorithm}[H]
\caption{ElGamal 公钥密码体制加密算法}
\KwInput{整数形式的消息 $ m $,公钥 $ P $}
\KwOutput{椭圆曲线上的点 $ x_1 $,整数 $ c $}
\KwData{椭圆曲线的参数 $ a, b $,它的一个生成元 $ G $}

\If{$ p >= n $ or $ p < 0 $}{
    抛出参数错误
}
\While{true}{
    $ k \leftarrow [1, n - 1] $ 中的随机数
    
    $ x_1 \leftarrow k G $
    
    $ x_2 \leftarrow k P $
    
    \If{$ x_2 $ 的横坐标 $ = 0 $}{
        \Continue
    }
}

$ c \leftarrow p x_2 $ 的横坐标

\Return{$ x_1, c $}
\end{algorithm}

\begin{algorithm}[H]
\caption{ElGamal 公钥密码体制解密算法}
\KwInput{密文对 $ (x_1, c $,私钥 $ n $}
\KwOutput{椭圆曲线上的点 $ x_1 $,整数 $ c $}
\KwData{椭圆曲线的参数 $ a, b $,它的一个生成元 $ G $}

\If{$ c \ge n $ or $ c < 0 $}{
    抛出解密出错
}
$ x_2 = n x_1 $

$ x'_2 \leftarrow x_2 $ 的横坐标

$ m = c x'_2 $

\Return{$ m $}
\end{algorithm}

\newpage

\subsection*{分析}

由于数据的上界已知, 各子算法的时空复杂度也已知, 所以时空复杂度为 $ \mathrm{O}(1) $. 

\subsection*{测试}

由于测试要求使用文件, 所以采用了文件加解密方式进行测试. 经过测试, 恢复的明文文件和原明文文件相同. 测试模块为 \verb|elgamal_test.py|.

\subsection*{优化}

\subsubsection*{主要算法}

其实对主要算法, 我也没想出很好的方式来优化. 但是实际上同样也可以通过优化子算法来优化算法的效率. 

\subsubsection*{文件加解密算法}

对文件加解密算法, 可以利用处理器的并行性来优化, 也可以提前算出需要的中间值. 

\section*{SM2 公钥密码体制}

\subsection*{原理}

SM2 公钥密码体制也是基于椭圆曲线上的离散对数问题的。但是,它有两点非常独特,一个是通过一个密钥生成函数(KDF)来生成密钥,另一个是它通过哈希函数实现了消息鉴别功能,提高了敌手伪造消息的成本。

SM2 算法需要使用国家密码管理局批准的哈希算法,这里使用的是 SM3 算法。事实上,SM2 密码体制支持多种有限域及多种点的表示,这里只实现了 $ F_p $ 域上点的未压缩表示。

\subsection*{伪代码}

\begin{algorithm}[H]
\caption{SM2 公钥密码体制密钥生成算法}
\KwInput{无}
\KwOutput{有限域上的点(私钥)$ d $,椭圆曲线上的点(公钥)$ P $}
\KwData{椭圆曲线的参数 $ a, b $,它的生成元 $ G $}

$ d \leftarrow [1, n - 2] $ 中的随机数

$ P \leftarrow d G $

\Return{$ d, P $}
\end{algorithm}

\begin{algorithm}[H]
\caption{SM2 公钥密码体制密钥派生函数(KDF)}
\KwInput{字节串 $ Z $,密钥长度 $ kLen $}
\KwOutput{密钥字节串 $ K $}
\KwData{哈希函数的输出位长度 $ v $}

$ ct $ 为一个 32 位的计数器变量

$ ct \leftarrow \mathtt{0x00000001} $

\For{$ i $ = 1 ... $ \lceil kLen / v \rceil $}{
    $ {Ha}_i \leftarrow Hash(Z \parallel ct $
    
    $ ct \leftarrow ct + 1 $
}

\If{$ kLen \mod v = 0 $}{
    $ {Ha}_{\lceil kLen / v \rceil} \leftarrow {Ha}_{\lceil kLen / v \rceil} $ 最左边的 $ kLen - (v \lfloor kLen / v \rfloor) $ 位}

$ K \leftarrow {Ha}_1 \parallel {Ha}_2 \parallel \cdots \parallel {Ha}_{\lceil kLen / v \rceil} $

\Return{$ K $}
\end{algorithm}

\newpage

\begin{algorithm}[H]
\caption{SM2 公钥密码体制加密算法}
\KwInput{明文 $ m $,公钥 $ P $}
\KwOutput{密文 $ c $}
\KwData{椭圆曲线的生成元 $ G $}

\While{true}{
    $ k \leftarrow [1, n - 1] $ 中的随机数
    
    $ c_1 \leftarrow k G $
    
    把 $ c_1 $ 转换为字节串
    
    $ h \leftarrow 1 $
    
    $ s \leftarrow h P $
    
    \If{$ s = P_\infty $}{
        抛出错误
    }
    
    $ x_2, y_2 \leftarrow k s $ 的横坐标和纵坐标
    
    把 $ x_2, y_2 $ 转换为字节串
    
    $ kLen \leftarrow m $ 的位长度
    
    $ t \leftarrow KDF(x_2 \parallel y_2, kLen) $
    
    \If{$ t $ 非全 $ 0 $ 的字节串}{
        \Break
    }
    
    $ c_2 \leftarrow m \oplus t $
    
    $ c_3 \leftarrow Hash(x_2 \parallel m \parallel y_2) $
    
    \Return{$ c_1 \parallel c_2 \parallel c_3 $}
}
\end{algorithm}

\begin{algorithm}[H]
\caption{SM2 公钥密码体制解密算法}
\KwInput{密文 $ c $,私钥 $ d $}
\KwOutput{明文 $ m $}
\KwData{椭圆曲线参数 $ a, b $,椭圆曲线的有限域 $ F_p $ 的模数 $ p $}

从 $ c $ 中提取 $ c_1, c_2, c_3 $

\If{按照格式出现索引错误}{
    抛出错误
}

\If{$ c_1 $ 不在椭圆曲线上}{
    抛出错误
}

$ x_2, y_2 \leftarrow d c_1 $ 的横坐标和纵坐标

把 $ x_2, y_2 $ 转换成字节串

$ kLen \leftarrow c_2 $ 的位长度

$ t \leftarrow KDF(x_2 \parallel y_2, kLen) $

\If{$ t $ 为全为 $ 0 $ 的字节串}{
    抛出错误
}

$ m \leftarrow c_2 \oplus t $

$ u \leftarrow Hash(x_2 \parallel m \parallel y_2) $

\If{$ u \neq c_3 $}{
    抛出错误
}

\Return{$ m $}
\end{algorithm}

\newpage

\subsection*{分析}

设消息的长度为 $ m $。

\subsubsection*{密钥生成算法}

该算法步骤恒定,因此时空复杂度都是 $ \mathrm{O}(1) $。

\subsubsection*{加密算法}

要注意的一点是,实际上 KDF 是需要不停地保存中间值的,这些中间值的长度跟 $ m $ 成正比,计算它们的时间也是。同样地,计算哈希值也要考虑输入数据的长度。因此,算法的总时空复杂度分别是 $ \mathrm{O}(m) $ 和 $ \mathrm{O}(m) $.

\subsubsection*{解密算法}

类似地,算法的总时空复杂度分别是 $ \mathrm{O}(m) $ 和 $ \mathrm{O}(m) $。

\subsection*{测试}

采用自动化脚本进行测试,脚本文件名为 \verb|sm2_test.py|。它可以测试加密和解密算法的正确性,以及自动加解密一个文件作为测试。经过测试,加密和解密算法互为逆,解密的文件也与加密的文件相同,因此算法正确。

\subsection*{优化}

\subsubsection*{密钥派生函数}

实际上,哈希函数的值可以并行计算,因为数据以来并不高。其实,结果 $ K $ 的值同样可以并行。这种并行对多核 CPU 更为有利,因为不同的 $ ct $ 值触发的代码执行路径一般是不相同的,在一个 CPU 上,分支预测器容易迷惑。而且,并行更有利于利用每个 CPU 的局部性。

\subsubsection*{加密算法和解密算法}

这两个算法实际上可以优化数据结构,这样可以简化对点的转换。但是并行性不够好,算法的分支情况也不大好预测。不过局部性还是比较强的。其实可以用上比较新的 x86 指令集,这样能直接支持 256 位整数运算。

\section*{总结}

\subsection*{椭圆曲线相关算法}

这些算法主要是对书上的公式的理解, 以及区分好各种特殊情况, 尤其是关于 $ P_{\infty} $ 的情况. 还有一种特殊情况值得注意: 自己与自己相加(或者说乘以 $ 2 $). 

\subsection*{椭圆曲线上的 Diffie-Hellman 密钥交换协议}

这一部分也主要是对书上公式的理解. 但是这一部分主要是一个协议原语, 所以能实现的主要是一方, 但测试时要测试双方, 这一点要注意. 

\subsection*{SM2 公钥密码体制}

这个公钥密码体制其实是椭圆曲线密码体制的更复杂的应用,其实更有意思。SM2 公钥密码体制实际上更为实用,因为它加上了消息鉴别的功能。

\subsection*{椭圆曲线上的 ElGamal 公钥密码体制}

这部分其实是以 Diffie-Hallman 密钥交换协议作为基础的. 其实该密码体制最重要的是它的随机性和密文扩张, 所以在与文件配合时要注意, 保存的文件会更大, 而且具有随机性. 

\subsection*{算法评估与优化}

对这次实验里的算法, 我其实没有想到太好的优化方法. 但是, 我感觉真正的优化应该去掉抽象, 把底层的椭圆曲线算法和顶层的算法结合起来, 虽然这只是一个直觉. 

\subsection*{系统设计与维护}

这次实验的完成, 实际上多少考验着系统设计能力. 椭圆曲线上点的算法, 实际上可以从点坐标所在的数域中抽象出来, 是点坐标的四则运算. 这时, 就可以使用运算符重载的方法, 来把对点坐标的运算抽象地写出来. 

同时, 对于下层的 $ F_p $ 中的元素, 肯定要抽象成类. 把 $ p $ 固定, 会使得该类并不能可移植, 因此, 一种解决办法是给一个函数, 输入 $ p $, 返回表示 $ F_p $ 中元素的类. 这样就类似于元类(实例是类的类), 不过这里是生成类的函数. 

对于椭圆曲线的各种参数, 可以把参数都封装到一个单独的模块里. 这里当然也得在那个模块中声明椭圆曲线对应的 $ F_p $. 声明完以后, 关于椭圆曲线的密码学原语只要导入对应的模块, 就能使用相应的椭圆曲线了. 

对文件的 ECB 模式加解密, 也可以考虑把负责 ECB 模式的代码抽象出来, 形成一个函数. 输入的时候把输入和输出文件名、分块加密或解密函数、加密或解密时相应分块的长度当做函数参数输入进去, 负责 ECB 模式的函数就可以执行对文件的加解密了. 这样能减小重复代码的量, 也有利于代码的可移植性, 更好地实现和测试其它加密算法. 

最终, 负责加解密的上层代码基本上不超过 20 行, 而且跟课件上的说明差不多是一致的. 这样极大地方便了开发, 也明显降低了 bug 发生的概率. 感觉这个实验其实比较需要总体的系统设计, 因为我想了两个小时, 才把大致的系统构架图想明白, 之后才开始编程. 

\subsection*{对课程的建议}

感觉这次密码学实验课程终于回到正轨了, 不再出难死人的大整数题了……这次我感觉思路比较正, 就是考对课上内容的理解和熟练应用, 以及系统的合理设计, 其实并没有什么太刁钻或太细节的地方. 感觉以后密码学实验其实应该把系统设计的良好程度放入总评, 这样也对得起我们想出良好的系统结构的努力. 

\subsection*{总结}

这次实验感觉路子比较正, 也很能考验系统设计能力. 当然, 我也感觉很有意思. 但是, 我感觉非常可惜的一点, 是我白白花了两个小时却找不到合适的资料. 所以希望以后老师或者助教学长们能帮我们找一下, 哪怕找最必需的也好. 

\end{document}
