\documentclass[12pt,a4paper]{article}

\usepackage{fontspec}

\usepackage{ctex}

\usepackage{amsmath}
\usepackage{amsfonts}
\usepackage{amssymb}

\usepackage{indentfirst}
\usepackage[ruled]{algorithm2e}
\usepackage{graphviz}

\setmainfont{Roboto}
\setmonofont{Noto Sans Mono CJK SC Regular}
\setCJKmainfont{Noto Sans CJK SC Regular}
\setmathrm{Latin Modern Math}
\setmathtt{Latin Modern Mono}

\renewcommand{\algorithmcfname}{算法}

\SetKw{Continue}{continue}
\SetKw{Break}{break}

\SetKwInOut{KwInput}{输入}
\SetKwInOut{KwOutput}{输出}
\SetKwInOut{KwData}{数据}
\SetKwData{KwResult}{结果}

\zihao{5}



\usepackage{listings}

\lstset{basicstyle=\ttfamily, language=Python}

\begin{document}

{
\zihao{2}
\begin{center}
密码学第九次实验报告
\end{center}
}

\section*{SHA-1 哈希算法}

\subsection*{原理}

SHA-1 哈希算法是比较经典的一种哈希算法, 适用于数据的验证等需求. 哈希算法是单向算法, 给定原来的数据, 能够产生长度固定的哈希值. 

SHA-1 哈希算法采用了 Merkle-Damgård 结构, 这种结构比较简单, 在理论上也能证明: 只要压缩函数 $ f $ 满足无碰撞性, 整个哈希函数也满足无碰撞性. 首先把消息按一定的规则进行填充, 然后把填充后的分组按顺序放入 $ f $ 函数, 输出作为下一轮执行 $ f $ 函数的参数之一. 第一轮没有上一轮执行 $ f $ 函数的结果, 就用初始向量 $ IV $ 代替. 其实, SHA-1 和 MD5 也比较相似. 

SHA-1 哈希算法的 $ f $ 函数是通过不同的混淆和置换方式, 来达到相对安全的消息杂凑的. 其中有模 $ \mathrm{2^{32}} $ 加法、按位运算等方式. 

实际上, SHA-1 已经能够相对容易地产生碰撞了, 并且由于硬件计算能力的日益强大和算法的进步, 攻击正在变得越来越现实. 因此, SHA-1 已经可以认为被废弃了. 

\subsection*{伪代码}

\begin{algorithm}[H]
\caption{SHA-1 哈希算法 $ f $ 函数}
\KwInput{数据块 $ b $,上次的结果 $ l $}
\KwOutput{结果 $ r $}

\For{$ i = $ 0 ... 15}{
    $ w_i \leftarrow b $ 的第 $ 4 i $ 个字节到第 $ 4 i + 3 $ 个字节按大端序转换成的整数
}

\For{$ i = $ 16 ... 79}{
    $ w_i \leftarrow (w_{i - 3} \oplus w_{i - 8} \oplus w_{i - 14} \oplus w_{i - 16}) <<< 1 $
}

\For{$ i = $ 0 ... 4}{
    $ h_i \leftarrow l $ 的第 $ 4 i $ 个字节到第 $ 4 i + 3 $ 个字节按大端序转换成的整数
}

$ a, b, c, d, e \leftarrow h_0, h_1, h_2, h_3, h_4 $

\For{$ i = $ 0 ... 79}{
    \If{$ 0 \le i \le 19 $}{
        $ f \leftarrow (b \wedge c) \vee ((\neg b) \wedge d) $
        
        $ k \leftarrow \mathtt{0x5a827999} $
    }
    
    \If{$ 0 \le i \le 39 $}{
        $ f \leftarrow b \oplus c \oplus d $
        
        $ k \leftarrow \mathtt{0x6ed9eba1} $
    }
    
    \If{$ 40 \le i \le 59 $}{
        $ f \leftarrow (b \wedge c) \vee (b \wedge d) \vee (c \wedge d) $
        
        $ k \leftarrow \mathtt{0x8f1bbcdc} $
    }
    
    \If{$ 60 \le i \le 79$}{
        $ f \leftarrow b \oplus c \oplus d $
        
        $ k \leftarrow \mathtt{0xca62c1d6} $
    }
    
    $ e, d, c, b, a \leftarrow d, c, b <<< 30, a, (a <<< 5) + f + e + k + w_i $
}

$ h_0, h_1, h_2, h_3, h_4 \leftarrow h_0 + a, h_1 + b, h_2 + c, h_3 + d, h_4 + e $

\Return{各 $ h_i $ 按大端序转换成 4 个字节再按下标顺序拼接的字符串}
\end{algorithm}

\begin{algorithm}[H]
\caption{SHA-1 哈希函数消息分组函数}
\KwInput{消息 $ m $}
\KwOutput{填充后的消息 $ m' $}

$ m' \leftarrow m $ 后面附加一个比特位 $ 1 $

$ m' $ 后面不断填充比特位 $ 0 $,直到其长度 $ l $(以位为单位)满足 $ l \equiv 448 (\mathrm{mod} ~ 512) $

$ m' $ 后面增加以 64 位大端序整数表示的 $ l $
\end{algorithm}

\begin{algorithm}[H]
\caption{Merkle–Damgård 结构哈希算法}
\KwInput{初始向量 $ {iv} $,填充后的消息 $ m' $,哈希函数的 $ f $ 函数 $ f({iv}, m') $}
\KwOutput{哈希值 $ h $}

$ {result} \leftarrow {iv} $

\For{$ b = m' $ 的每一块}{
    $ {result} \leftarrow f(result, b) $
}

\Return{$ {result} $}
\end{algorithm}

\begin{algorithm}[H]
\caption{SHA-1 算法}
\KwInput{消息 $ m $}
\KwData{SHA-1 算法的初始向量 $ {iv} $}

$ m' \leftarrow m $ 填充后的消息

$ {result} \leftarrow $ Merkle–Damgård 结构哈希算法 $ ({iv}, m', f) $

\Return{$ {result} $}
\end{algorithm}

\subsubsection*{HMAC 算法}

\begin{algorithm}[H]
\caption{SHA-1 HMAC 算法}
\KwInput{消息 $ m $,密钥 $ k $,哈希函数 $ f(m) $,它的输出字节数 $ l_0 $}
\KwData{两个掩码字节串 $ ipad, opad $}

$ l \leftarrow k $ 的字节数

\If{$ l > l_0 $}{
    $ k \leftarrow f(k) $
}

$ k \leftarrow k \parallel l_0 - l $ 个 $ 0 $ 字节

$ k_1 \leftarrow k \oplus ipad $

$ H_1 \leftarrow f(k_1 \parallel m) $

$ k_2 \leftarrow k \oplus opad $

$ H_2 \leftarrow f(k_2 \parallel H_1) $

\Return{$ H_2 $}
\end{algorithm}

\subsubsection*{SHA-1 HMAC 算法}

\begin{algorithm}[H]
\caption{SHA-1 HMAC 算法}
\KwInput{消息 $ m $,密钥 $ k $}
\KwData{SHA-1 哈希函数 $ f $,SHA-1 哈希函数的输出字节数 $ l_0 $}

\Return{HMAC 算法 $ (m, k, f, l_0) $}
\end{algorithm}

\subsection*{分析}

以下设消息长度为 $ l $. 

\subsubsection*{$ f $ 函数}

由于 $ f $ 函数的输入规模和计算语句的执行次数恒定, 所以它的时空复杂度都是 $ \mathrm{O}(1) $.

\subsubsection*{消息分组函数}

消息分组函数附加的数据长度是有上限的, 但是产生的分组与长度大致成正比关系. 因此易知它的时间复杂度为 $ \mathrm{O}(l) $. 由于需要的空间上限是固定的(最多 512 位), 空间复杂度为 $ \mathrm{O}(1) $. 

\subsubsection*{Merkle–Damgård 结构哈希算法}

该结构的哈希算法的时间复杂度由 $ f $ 函数决定. 这里 $ f $ 函数时空复杂度都是 $ \mathrm{O}(1) $. 但是分组有 $ \mathrm{O}(l) $ 个, 所以时间复杂度是 $ \mathrm{O}(l) $. 由于数据之间没有依赖, 而且 $ f $ 函数的空间复杂度是 $ \mathrm{O}(1) $, 所以空间复杂度是 $ \mathrm{O}(1) $. 

\subsubsection*{SHA-1 算法}

该算法就是 Merkle–Damgård 结构哈希算法的一个封装, 所以时间复杂度和空间复杂度也分别是 $ \mathrm{O}(l) $ 和 $ \mathrm{O}(1) $. 

\subsubsection*{HMAC 算法}

设密钥的长度为 $ l_K $. 

首先若密钥长度超过相应哈希算法的分组长度, 则密钥的哈希值就要被计算出来. 所以这里的时间复杂度是 $ \mathrm{O}(l_K) $, 空间复杂度是 $ \mathrm{O}(1) $. 之后, 密钥的长度就固定了. 对密钥的填充和与 $ ipad $ 或 $ opad $ 的异或时空复杂度都是 $ \mathrm{O}(1) $. 

然后对消息和密钥拼接, 再求哈希值. 这里密钥经过处理后, 长度恒定, 所以这步的时间复杂度是 $ \mathrm{O}(l) $, 空间复杂度是 $ O(1) $. 

之后的步骤数据规模恒定, 需要的时间和空间也恒定, 所以时空复杂度都是 $ \mathrm{O}(1) $. 

所以算法总的时空复杂度分别是 $ \mathrm{O}(l_K + l) $ 和 $ \mathrm{O}(1) $. 

\subsubsection*{SHA-1 HMAC 算法}

该算法是 HMAC 算法的一个封装, 所以时空复杂度和它相同, 也都分别是 $ \mathrm{O}(l_K + l) $ 和 $ \mathrm{O}(1) $. 

\subsection*{优化}

\subsubsection*{$ f $ 函数}

$ i \in [0, 19] $ 时的逻辑函数是按位选择函数, $ i \in [40, 59] $ 时的逻辑函数是按位取多数函数. 这些函数都有多种变形, 有的利于通用处理器实现, 有的利于专用电路实现. 

$ w_i $ 的计算在第 32 - 79 轮时可以优化成 $ w_i = (w_{i - 6} \oplus w_{i - 16} \oplus w_{i - 28} \oplus w_{i - 32}) <<< 2 $. 这种变换可以使得各操作保持 64 字节对齐, 并且把 $ w_i $ 向 $ w_{i - 3} $ 的依赖去掉了, 更有利于 SIMD 等向量指令集实现 SHA-1. 

\subsubsection*{消息分组函数}

消息分组函数可以计算出要填充的消息末尾后, 用 Python 的迭代器实现, 只要迭代到消息末尾后, 再追加即可. 这样可以减少中间结果的内存占用, 也可以在 HMAC 算法中复用. 

\subsubsection*{Merkle-Damgård 结构哈希算法和 SHA-1 算法}

这种算法虽然安全性较高, 但是好像数据依赖性也较高, 相对难以在并行性上优化. 但是可以通过优化 $ f $ 函数, 间接优化哈希算法. 

\subsubsection*{HMAC 算法和 SHA-1 HMAC 算法}

由于 $ K^+ \oplus ipad $ 和 $ K^+ \oplus opad $ 是可以提前计算出来的, 而且如果下层的哈希算法是 Merkle-Damgård 结构的话, 可以提前把这块放入 $ f $ 函数计算结果, 并且把结果当作相应的新初始向量 $ iv' $, 所以 SHA-1 HMAC 算法可以这样优化. 同样地, 并行性也是一个问题, 但是提前计算出新的 $ iv' $ 能够稍微提高并行性, 而且能够提高计算短消息的 HMAC 的速度. 

\subsection*{测试}

采用 \verb|sha1_test.py| 进行自动测试. 该文件大致上是选择 10000 个随机字节串作为消息, 10000 个随机字节串作为密钥, 把求出的结果与标准库中的 SHA-1 和 HMAC 算法进行比较. 如果有不同, 就打印出产生错误的消息和密钥, 否则打印出“\verb|test passed|”并退出. 

运行了该文件多次, 都能通过测试. 所以可以认为算法没有问题. 

\section*{Hash 函数生日攻击}

\subsection*{原理}

对 Hash 函数的生日攻击是把 Hash 函数看成输入随机输出也随机但对某个确定输入输出确定的函数. 这种假设也符合理想 Hash 函数的性质. 这里的生日攻击是找出一对消息 $ x $ 和 $ y $, $ s.t. H(x) = H(y) $. 其中 $ H(m) $ 为哈希函数. 

可以通过概率论的知识得到, 若 Hash 函数是理想 Hash 函数, 攻击的代价是大约 $ 2^{\frac{n}{2}} $ 次 Hash 运算, 其中 $ n $ 是 Hash 函数结果的二进制位数. 

\subsection*{伪代码}

\begin{algorithm}[H]
\caption{哈希函数生日攻击}
\KwInput{随机字节串生成函数 $ r() $,哈希函数 $ H(m) $}
\KwOutput{一对字节串 $ x_1, x_2 $}

$ x_1, x_2 \leftarrow r(), r() $

\While{$ H(x_1) \neq H(x_2) $ or $ x_1 = x_2 $}{
    $ x_1, x_2 \leftarrow r(), r() $
}

\Return{$ x_1, x_2 $}
\end{algorithm}

\subsection*{分析}

由于攻击的代价是大约 $ 2^{\frac{n}{2}} $ 次 Hash 运算, 而且每次 Hash 运算的时空复杂度都是 $ \mathrm{O}(1) $, 再加上每次 Hash 计算时没有数据依赖, 所以整个算法的时间复杂度为 $ \mathrm{O}(2^\frac{n}{2}) $, 空间复杂度为 $ \mathrm{O}(1) $. 

\subsection*{优化}

其实该算法相当好并行, 所以可以并行计算, 这样可以线性地提高效率. 其实也可以先计算短消息, 再不断延长, 这样可以直接把已经算好的哈希作为新的初始向量, 提高计算效率. 

实际上, 对已经有的哈希算法采用这种攻击不现实, 因为$ n $ 太大了. 在真正测试的时候, 采用了截断的哈希函数, 也就是把消息经过 SHA-1 哈希算法的结果取前两个字节. 这样 $ n = 16 $, 能够保证在可行的时间内找到一对哈希值相同的消息. 

\subsection*{测试}

测试采用自动化测试, 文件为 \verb|hash_collision_test.py|. 文件会自动找出一对有冲突的消息, 并输出它们的内容和冲突的哈希值. 这样实际上就起到了对算法进行测试的作用. 经过检验, 找到的几对有冲突的消息确实是有冲突. 

\section*{英文消息变形生成算法}

\subsection*{原理}

对英文消息找出消息变形,有两种比较实际的方式:根据语义的方式和插入“空格—退格”对的方式。但是,根据语义的方式其实比较难,因为难以编写确定性算法。因此,采用了插入“空格—退格”对的方式。

插入“空格—退格”对的方式比较容易实现,但是一般是在空格附近插入“空格—退格”对。一般来说,需要用一种方式遍历原来消息中的空格,而且要注意消息中的空格数目应该是变化的。有一种可行的方式是用一个自然数表示,这样想的话,就变成了一个数学问题。设原来消息的空格数为 $ n $,那么由于对每个空格前面(不妨设在每个空格前面插入“空格—退格”对)都可以插入自然数(这里也包括 0)个“空格—退格”对,所以这变成了找到一种从 $ \mathbf{N} $ 到 $ \mathbf{N}^n $ 的映射。可以证出 $ \mathbf{N}^n $ 可数,即其势与 $ \mathbf{N} $ 相等。因此双射应该是能找到的。但是,考虑到实现的难易程度,这里只是构造了一个(或者说一系列)普通映射。

该映射如下:

\begin{gather*}
f_n: \mathbf{N} \to \mathbf{N}^n \\ 
x \to (a_0, a_1, \cdots, a_{n - 1})
\end{gather*}

其中 $ a_{x \mod n} = \lfloor \frac{x}{n} \rfloor + 1 $,其它各 $ a_i = \lfloor \frac{x}{n} \rfloor $。这样,就能利用这个映射实现能够意识到原来数据空格个数的消息变形,效果比较好。

由于需要变形的消息是英文消息,所以把英文消息表示成使用 8 位 ASCII 码的字节串形式。这样方便解释算法,也方便实现。

\subsection*{伪代码}

\begin{algorithm}[H]
\caption{英文消息变形生成算法}
\KwInput{消息 $ m $,要得到的变形数目 $ n $}
\KwOutput{消息 $ m $ 的 $ n $ 个变形组成的列表(按一定顺序排列)}
\KwData{上一节中提到的映射 $ f $}

$ {result} \leftarrow $ 空列表

$ c \leftarrow m $ 中的空格数目

\If{$ c = 0 $}{
    抛出错误
}

\For{$ i = $ 1 ... $ n $}{
    $ \vec{a} \leftarrow f_c(i) $
    
    \For{$ j = $ 1 ... $ c $}{
        $ {curr} \leftarrow $ 在 $ m $ 中的第 $ j $ 个空格前面加入 $ a_j $ 个“空格—退格”对得到的结果
    }
    
    在 $ {result} $ 后面添加 $ {curr} $
}
\end{algorithm}

\subsection*{分析}

设原消息长度为 $ l $ 字节,空格个数为 $ n $,要生成的变形个数为 $ c $。

首先,根据 $ f_n(x) $ 映射的性质,能知道最终列表中每个消息的长度不会超过 $ l + 2 \lceil \frac{c}{n} \rceil $ 字节。而且,最长的那些消息的长度也是这个数量级。然后,要生成的变形个数为 $ c $,因此算法的总空间复杂度为 $ \mathrm{O}(c l + \frac{c^2}{n}) $。

然后,对消息遍历一遍来找空格个数的时间复杂度为 $ \mathrm{O}(l) $。对每个消息,插入空格的时间复杂度为 $ \mathrm{O}(n) $。但是一共有 $ c $ 个消息,而且对每个变形消息,隐含拷贝消息的步骤,每次拷贝时间复杂度为 $ \mathrm{O}(l + n) $。因此,算法的总时间复杂度为 $ \mathrm{O}(c l + c n) $。

\subsection*{优化}

首先很容易想到的一点,就是优化插入空格。对插入空格的优化,可以通过链表来进行,因为链表是一种插入时时空复杂度都是 $ \mathrm{O}(1) $ 的线性表。而且,只要边读边检测空格时再加上建立链表的步骤,建立链表的代价就不大重要了。

然后,可以来优化映射,把插入得少的一些位用上,这样来优化时空复杂度。首先,$ \mathbf{N} $ 到 $ \mathbf{N}^2 $ 的双射可以利用对角线方法得到,因此更高维度的双射,可能也可以用类似方法得到。

事实上,其实还有其它同学设计的算法(或者也可以说映射)比这个映射更好,在这里就不赘述了。

\subsection*{测试}

采用自动化脚本进行测试。测试脚本文件名为 \verb|msg_permutation_test| \verb|.py|。脚本主要通过一个英文短语,测试能否得到消息变形列表,以及该列表的正确性。经过测试,发现能够得到消息列表,该列表也符合前面定义的规则。所以可以说,实现一般是正确的。

\section*{SHA-3 哈希算法}

\subsection*{原理}

SHA-3 哈希算法是为了取代 SHA-2 哈希算法而被创造的,它能提供比 SHA-2 更强的安全性。它的核心并不是分组链接结构,而是海绵结构,这种结构是一种全新的结构,它能够提供更高的灵活性。类似 AES,SHA-3 哈希算法也是被公开评选出来的,这使得加密算法的选择更为透明,也比较能避免算法中的后门。

被公开评选出来的 SHA-3 哈希算法是 Keccak 系列密码算法中的一种,它由同名的 Keccak 团队发明出来,经过了比较先进的安全强度检验。Keccak 系列密码算法的核心是 $ f $ 函数,它也决定了 Keccak 系列算法中间状态的大小。中间状态是三维的、分行(row)、列(column)和纵(lane)。其中行和列的数量固定为 5。它的核心是五个按顺序执行的变换:$ \theta, \rho, \pi, \chi, \iota $ 变换。它们比较完善地践行了对称密码体制的两个基本变换:代替和置换。它们有的负责一个纵内的变换,有的负责行和列的变换;有的负责进行线性变换,有的负责进行非线性变换;有的负责改变纵的位置,有的负责改变纵的内容;有的保持对称性,有的引入常数来打破对称性。综合来看,这是一种理论坚实、算法完善、经得起检验的密码算法。而且更为可贵的是,它的标准是开放的,Keccak 团队甚至在他们的网站(\texttt{https://keccak.team})上公布了标准。

SHA-3 哈希算法就是在 $ f $ 函数上进行的。但是,它用到的是海绵结构。海绵结构有两个阶段:吸收阶段吸收输入的信息,挤压阶段处理(也可以看成放出)输入的信息。通过保持 $ f $ 函数的输入比特率和容纳量之和不变(因此 SHA-3 系列哈希算法只用到一种 $ f $ 函数),更改输入比特率和输出长度,SHA-3 哈希算法能做到输出可变长度的哈希值,在这方面做到了在各个方面能够替代原先的 SHA 算法的要求。

这里用到的 SHA-3 哈希算法,是遵从 Keccak 团队官方网站上的标准的。这样,能够调用 \texttt{hashlib} 库进行自动化测试。

值得注意的一点是,在把状态打包成字节时,应该按照列优先的方式进行打包,而且把每个纵按照小端序的方式进行打包。在解包时,也要做与其对应的操作。这样才能构造出符合正确的 SHA-3 算法。这样,也能把五个原来基于位变换的操作,转换成基于整个纵的变换,提高了效率。

基于 SHA-3 的 HMAC 算法和普通的 HMAC 算法没有什么区别,不过按照标准,每块的长度需要指定。因此,优化会麻烦一些。

\subsection*{伪代码}

\begin{algorithm}[H]
\caption{SHA-3 哈希算法 $ f $ 函数}
\KwInput{200 个字节组成的字节串 $ b $}
\KwOutput{变换后的字节串 $ b' $}
\KwData{循环移位位数表 $ r[0 \cdots 4][0 \cdots 4] $,轮常数 $ {rc}[0 \cdots 23] $}

把 $ b $ 按照前面的方式转换,得到 64 位整数数组 $ A[0 \cdots 4][0 \cdots 4] $

\For{$ i = $ 0 ... 23}{
    \For{$ x = $ 0 ... 4}{
        $ C[x] \leftarrow A[x, 0] \oplus A[x, 1] \oplus A[x, 2] \oplus A[x, 3] \oplus A[x, 4] $
    }
    
    \For{$ x = $ 0 ... 4}{
        $ D[x] \leftarrow C[(x - 1) \mod 5] \oplus (C[(x + 1) \mod 5] <<< 1)$
    }
    
    \For{$ x = $ 0 ... 4}{
        \For{$ y = $ 0 ... 4}{
            $ A[x, y] \leftarrow A[x, y] \oplus D[x] $
        }
    }
    
    \For{$ x = $ 0 ... 4}{
        \For{$ y = $ 0 ... 4}{
            $ B[y][(2 x + 3 y) \mod 5] \leftarrow A[x][y] <<< r[x][y] $
        }
    }
    
    \For{$ x = $ 0 ... 4}{
        \For{$ y = $ 0 ... 4}{
            $ A[x][y] \leftarrow B[x][y] \oplus ((\neg B[(x + 1) \mod 5][y]) \wedge B[(x + 2) \mod 5][y]) $
        }
    }
    $ A[0][0] \leftarrow A[0][0] \oplus {rc}[i] $
    
    }

\Return{$ A[0 \cdots 4][0 \cdots 4] $ 按照对应的方式转换回来的字节串 $ b' $
}
\end{algorithm}

\begin{algorithm}[H]
\caption{Keccak 哈希函数构造主算法}
\KwInput{消息 $ m $,比特速率 $ r $,1 个字节长的限定后缀 $ d $,输出长度(以字节为单位)$ l $}
\KwOutput{长度为 $ l $ 字节的哈希值 $ h $}
\KwData{纵长度 $ w $}

$ P \leftarrow m \parallel d $

在 $ P $ 后面附加 0 字节,使得其长度(以字节为单位)能被 $ \lfloor \frac{r}{8} \rfloor $ 整除

$ P $ 的最后一个字节 $ \leftarrow $ 它与 $ \mathtt{0x80} $ 异或后的结果

$ b \leftarrow $ 200 字节长的字节数组

\For{$ P $ 中每个 $ \lfloor \frac{r}{8} \rfloor $ 字节大小的块}{
    $ b $ 的前 $ \lfloor \frac{r}{8} \rfloor $ 个字节 $ \leftarrow $ 它与 该块逐字节异或后的结果
    
    $ b \leftarrow f(b) $
}

$ h \leftarrow $ 空字节串

\While{还需要更多的输出}{
    $ h \leftarrow h \parallel b $
    
    $ b \leftarrow f(b) $
}
\end{algorithm}

\begin{algorithm}[H]
\caption{SHA-3 哈希算法}
\KwInput{需要的哈希值位数 $ l $,消息 $ m $}
\KwOutput{长度为 $ \lfloor \frac{l}{8} \rfloor $ 字节的哈希值 $ h $}

\If{$ l = 224 $}{
    \Return{$ {Keccak}(m, 1152, \mathtt{0x06}, 224) $}
}

\If{$ l = 256 $}{
    \Return{$ {Keccak}(m, 1088, \mathtt{0x06}, 256) $}
}

\If{$ l = 384 $}{
    \Return{$ {Keccak}(m, 832, \mathtt{0x06}, 384) $}
}

\If{$ l = 512 $}{
    \Return{$ {Keccak}(m, 576, \mathtt{0x06}, 512) $}
}

抛出错误
\end{algorithm}

\begin{algorithm}[H]
\caption{SHA-3 HMAC 算法}
\KwInput{需要的 MAC 值位数 $ l $,消息 $ m $,密钥 $ k $}
\KwOutput{MAC 值 $ h $}
\KwData{基于哈希函数 $ f(m) $ 的 HMAC 算法 $ \mathrm{HMAC}(m, k, f, l_O) $, SHA-3 哈希函数 $ f(l, m) $}

\If{$ l = 224 $}{
    $ g(m) \leftarrow f(224, m) $
    
    \Return{$ \mathrm{HMAC}(m, k, g, 224) $}
}

\If{$ l = 256 $}{
    $ g(m) \leftarrow f(256, m) $
    
    \Return{$ \mathrm{HMAC}(m, k, g, 136) $}
}

\If{$ l = 384 $}{
    $ g(m) \leftarrow f(384, m) $
    
    \Return{$ \mathrm{HMAC}(m, k, g, 104) $}
}

\If{$ l = 512 $}{
    $ g(m) \leftarrow f(512, m) $
    
    \Return{$ \mathrm{HMAC}(m, k, g, 72) $}
}

抛出错误
\end{algorithm}

\subsection*{分析}

设消息长度为 $ l $,密钥长度为 $ l_k $,哈希算法的输出长度为 $ l_o $。

\subsubsection*{SHA-3 哈希算法 $ f $ 函数}

易知算法空间上界和计算上界都固定,因此它的时空复杂度都是 $ \mathrm{O}(1) $。

\subsubsection*{Keccak 哈希函数构造主算法}

首先,在初始化阶段,需要 $ \mathrm{O}(l) $ 量级的空间来存储 $ P $,但是生成 $ P $ 一般需要 $ \mathrm{O}(1) $ 的时间复杂度。然后,易知分块数目在 $ \mathrm{O}(l) $ 量级。这样,由于对每个分块都需要执行一次 $ f $ 函数,所以吸收阶段的时间复杂度为 $ \mathrm{O}(l) $,空间复杂度也为 $ \mathrm{O}(l) $。之后,在挤压阶段,易知每次输出的块长度固定。而每次挤压需要的时空复杂度都为 $ \mathrm{O}(1) $。因此,挤压阶段需要的时空复杂度都是 $ \mathrm{O}(l_o) $。算法总时空复杂度都是 $ \mathrm{O}(l + l_o) $。

\subsubsection*{SHA-3 哈希算法}

易知该算法就是 Keccak 哈希函数构造主算法的一个封装,因此时空复杂度与 Keccak 哈希函数构造主算法相同,都是 $ \mathrm{O}(l + l_o) $。

\subsubsection*{SHA-3 HMAC 算法}

易知该算法是 HMAC 算法的一个封装,而第一次求哈希值的时空复杂度根据上面和 HMAC 算法的步骤知这一步的时空复杂度都是 $ \mathrm{O}(l_k + l_o) $,因为第一次求哈希值可能还包括把密钥 $ k $ 再求一遍哈希值的步骤。之后,求第二遍哈希值时,时空复杂度由于指定的哈希值字节数给定,为 $ \mathrm{O}(1) $。因此,算法的总时空复杂度都是 $ \mathrm{O}(l_k + l_o) $。

\subsection*{优化}

\subsubsection*{SHA-3 哈希算法 $ f $ 函数}

实际上,这个函数的每一步变换,原来是基于位的,而且是把状态作为比特串做的运算。但是,根据位运算的性质,可以把每个纵看成一个整体,把它变成每个纵整体的位运算。这样既能提高效率,也能方便实现。

还有一点,就是 $ \rho $ 变换的移位次数不要现算,可以查表,因为现算太麻烦了。这样显然能提高效率,是一种以时间换空间的做法。

\subsubsection*{Keccak 哈希函数构造主算法}

该算法主要的优化点就是海绵结构。首先可以不实际上合并 $ m $ 和 $ d $,只建立对它们的引用即可,而且应该是只维护两项,这样能够省下空间。然后,可以只维护 $ P $ 的偏移,看到剩下的空间不足一个块了,就现场填充和异或在一起做(实际上也就不用填充太多了,因为填充数据是 0 字节),这样也能够省下空间,减少最后一次异或的个数。

最后,由于输出空间可以预知,所以可以提前分配,挤压阶段中得到每一步的输出,检查还剩多少字节,并把相应的结果写入已经分配好的连续空间内即可。这样能够利用内存访问的连续性提高效率,也能减少分支数量,间接提高效率。

\subsubsection*{SHA-3 哈希算法}

该算法主要的优化点是哈希所用参数的查找。当然,也可以通过某些数学关系,不需要查找,直接计算出需要的参数。不过,最容易实现的还是查找表。

\subsubsection*{SHA-3 HMAC 算法}

该算法相对其它 HMAC 算法的一个独特之处就是它把哈希函数的输出字节数改变了,而且都比原来相应哈希函数的输出字节数长。这样实现能够符合 Python 中 \texttt{hashlib} 的结果。因此,一个比较明显的优化点是优化第一块的计算。类似基于分组链接结构的哈希算法改变初始向量 $ {iv} $,这里也可以提前计算出状态 $ b $。不过由于长度的差异,优化时可能要注意特殊情况。

\subsection*{测试}

采用自动化脚本进行测试,脚本文件名为 \texttt{sha3\_test.py}。该脚本能够随机生成 500 对长度也随机的消息 $ m $ 和密钥 $ k $,然后对 SHA-3 的各种输出长度,把实现的 SHA-3 算法和 SHA-3 HMAC 算法的结果与 Python 中 \texttt{hashlib} 和 \texttt{hmac} 库的结果相比较。一旦有错误,测试脚本就会报错,打印出出现错误的消息 $ m $(在测试 HMAC 算法报错时也包括密钥 $ k $)。经过多次运行,没有发生错误,因此可以判断实现一般是正确的。

\section*{总结}

\subsection*{SHA-1 哈希算法}

这个算法是我第一次尝试写哈希算法. 这次写算法让我巩固了 Merkle-Damgård 结构的相关知识, 感觉更加明白了哈希的原理. 

HMAC 算法让我更加明白了安全协议和在原语上构建协议的重要性, 它用简单的哈希算法构造出了相对复杂的原语, 但是安全性同样相当高. 

\subsection*{哈希函数生日攻击}

哈希函数的生日攻击是对密码学函数的又一次攻击. 这次攻击我感觉很有意思, 能够用数学原理把攻击变成实用. 
  
\subsection*{英文消息变形生成算法}

这个算法主要是辅助生日攻击用的. 其实这个算法写起来比较有挑战性, 尤其是找映射, 既要考虑到消息本身的结构特征, 又要考虑可实现性. 因此, 这个算法还是比较有挑战性的. 

\subsection*{SHA-3 哈希算法}

这个算法是现在比较先进的公开算法,实现起来也比较有挑战性。这个算法给我的最大启示,是看文档一定要看好懂的,这样能够快速入门,尤其是概念比较艰深的那些。我看 FIPS 202,把我绕进去了。回过头来看 Keccak 团队的伪代码,很快就实现出来了。还有一个启示,就是涉及细节的时候,一定要实践,根据实践推测事半功倍。我在考虑字节跟纵数组的转换的时候,百思不得其解,后来看到 Keccak 团队的伪代码和网上一个一步一步求 SHA-3 的教程,终于明白转换方式了。这个算法给我最大的经验教训是不光要重视算法,还要重视工具层面。

\subsection*{算法分析与优化}

这次对算法的优化也是有相当重要的地位的. 比如对 SHA-1 函数的优化, 可以让它更适合在各种场景应用. 对 HMAC 算法的优化, 是基于对算法的深刻理解. 

同时, 对算法优化的经典思路仍然适用. 对英文消息的变形方案, 是基于数学. 对 SHA-3 函数的优化, 是基于看问题观点的转化. 

\subsection*{系统设计与维护}

这次实验也是需要一定的系统设计的. 在对哈希函数的设计中, 可以把算法的结构和具体的 $ f $ 函数分开. 在对 HMAC 算法的设计中, 可以把哈希函数的 $ f $ 函数设计成可以改变初始向量 $ iv $, 这样就可以方便优化. 在实现 SHA-3 哈希算法时, 要把层分清楚, 这样不但能够快速实现, 而且能够实现其它 Keccak 团队发明的算法, 因为 $ f $ 函数是共通的. 

\subsection*{对课程的建议}

感觉要求可以更贴近原理, 这样更好一些. 

\subsection*{总结}

这次哈希算法实验感觉不错, 让我更加明白了哈希算法的原理, 掌握了对哈希算法的初步攻击技巧. 我会努力面对以后的挑战. 

\end{document}
